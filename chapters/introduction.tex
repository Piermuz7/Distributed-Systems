\chapter{Introduction}\label{chap:intro}
% The remainder of the paper is structured as follows:

Distributed systems have become an integral part of modern computing, enabling the development of scalable, efficient, and resilient applications.
These systems consist of multiple autonomous computers that communicate and coordinate to achieve common goals.
One significant advancement in distributed systems is the advent of \gls{p2p} architectures, which eliminate the need for centralized control, allowing for greater scalability and fault tolerance.

\section{Evolution of Distributed Systems}
Historically, distributed systems were designed with a centralized approach, where a single server managed the operations and resources.
However, this model faced challenges such as single points of failure, scalability limitations, and bottlenecks.
The need for more resilient and scalable systems led to the development of decentralized architectures, where multiple nodes share control and resources (Tanenbaum \& Van Steen, 2007).

\section{Importance of \gls{p2p} Systems}

\gls{p2p} systems emerged as a solution to the limitations of centralized architectures.
In \gls{p2p} networks, each node can act as both a client and a server, facilitating direct resource sharing among peers.
This decentralized model offers numerous benefits, including improved scalability, fault tolerance, and resource utilization.
Applications of \gls{p2p} systems range from file sharing and content distribution to distributed computing and collaborative platforms (Androutsellis-Theotokis \& Spinellis, 2004).