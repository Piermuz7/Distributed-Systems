\chapter{Conclusion}\label{chap:conclusion}
The Chord framework represents a significant advancement in the design and implementation of \gls{p2p} systems.
Its robust and scalable architecture addresses key challenges in distributed systems, such as efficient resource discovery, fault tolerance, and dynamic node management.
By organizing nodes in a circular identifier space and utilizing finger tables for efficient routing, Chord ensures optimal performance even in large-scale networks.

Chord's influence extends beyond its original design, inspiring the development of various systems and applications in file sharing, distributed databases, content distribution, and distributed computing.
Its principles continue to shape the future of decentralized architectures, driving innovation and enhancing the capabilities of distributed systems.

As the field of distributed systems evolves, the Chord framework remains a foundational model for designing scalable, efficient, and resilient \gls{p2p} networks.
Its contributions to the field underscore the importance of decentralized approaches in achieving robust and high-performance distributed systems.