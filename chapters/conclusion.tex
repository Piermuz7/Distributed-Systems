\chapter{Conclusion}\label{chap:conclusion}
In this paper, we have covered the most important aspects of the Chord protocol, starting initially with the basics of distributed systems, and ending with \gls{p2p} systems. Of the latter, the two main architectures, unstructured and structured, were described, real case examples were given, and a comparison between the two was also provided.
Following this, we described the Chord protocol in its entirety, detailing the main components of the systems such as node identifiers, hashing functions, finger tables.
The Lookup mechanism was explained, also by means of an example, and the aspects of joining and leaving of nodes from the Chord network were also discussed.
Finally, all the advantages and limitations of the protocol were described. 
We also observed that over the years, the protocol has been used and inspired for the development of applications in the fields of file sharing systems, distributed databases, \gls{cdn}, distributed computing and sensor networks.
The Chord framework represents a significant advancement in the design and implementation of \gls{p2p} systems.
Its robust and scalable architecture addresses key challenges in distributed systems, such as efficient resource discovery, fault tolerance, and dynamic node management.
By organizing nodes in a circular identifier space and utilizing finger tables for efficient routing, Chord ensures optimal performance even in large-scale networks.
Its principles continue to shape the future of decentralized architectures, driving innovation and enhancing the capabilities of distributed systems.
As the field of distributed systems evolves, the Chord framework remains a foundational model for designing scalable, efficient, and resilient \gls{p2p} networks.
Its contributions to the field underscore the importance of decentralized approaches in achieving robust and high-performance distributed systems.
The continuous development and application of Chord in various domains highlight its enduring relevance and potential for future advancements in distributed computing.