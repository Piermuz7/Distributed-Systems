\chapter{Systems Based on Chord}
Chord's efficient and scalable architecture has inspired the development of various systems and applications in the realm of distributed computing and \gls{p2p} networks.

\section{File Sharing Systems}
One of the most common applications of Chord is in file sharing systems.
These systems leverage Chord's \gls{dht} to store and retrieve files efficiently.
Examples include:
\begin{itemize}
    \item \gls{cfs}: \gls{cfs} uses Chord to provide a robust and efficient file storage and retrieval service.
    It distributes file blocks across multiple nodes, ensuring redundancy and fault tolerance (Dabek et al., 2001).
    \item Ivy: Ivy is a multi-user read/write \gls{p2p} file system that uses Chord for efficient data location.
    It allows users to collaboratively manage and share files in a decentralized manner (Muthitacharoen et al., 2002).
\end{itemize}

\section{Distributed Databases}
Chord's principles have been applied to the design of distributed databases, providing scalable and fault-tolerant data storage solutions.
\begin{itemize}
    \item Dynamo: Developed by Amazon, Dynamo is a distributed key-value store that uses concepts similar to Chord for partitioning and replicating data across multiple nodes.
    It ensures high availability and durability, even in the presence of node failures (DeCandia et al., 2007).
    \item Cassandra: Cassandra is a highly scalable distributed database that employs a Chord-like \gls{dht} for data distribution.
    It provides a decentralized and fault-tolerant architecture, making it suitable for large-scale applications (Lakshman \& Malik, 2010).
\end{itemize}

\section{\acrlong{cdn}}

Chord's efficient resource discovery and routing capabilities make it a suitable choice for content distribution networks.
\begin{itemize}
    \item Coral: Coral is a content distribution network that uses Chord to locate and distribute web content efficiently.
    It reduces the load on origin servers and enhances the scalability of content delivery (Freedman et al., 2004).
    \item Vuze: Vuze, a popular BitTorrent client, utilizes Chord to manage its \gls{dht} for peer discovery and content distribution.
    It enables efficient sharing and downloading of large files in a decentralized manner.
\end{itemize}

\section{Distributed Computing}
Chord has also been used in distributed computing platforms to manage and allocate computational resources.
\begin{itemize}
    \item \acrshort{boinc}: The \gls{boinc} uses Chord-like \glspl{dht} to manage and distribute computational tasks across volunteer nodes.
    It supports large-scale scientific computing projects by leveraging the idle processing power of participant computers (Anderson, 2004).
    \item SETI@home: SETI@home, a distributed computing project aimed at analyzing radio signals for signs of extraterrestrial intelligence, employs Chord to efficiently distribute and manage data analysis tasks across a global network of volunteer computers (Anderson et al., 2002).
\end{itemize}

\section{Sensor Networks}
Chord has also been adapted for use in sensor networks, where efficient data aggregation and query processing are critical.
\begin{itemize}
    \item ChordGrid: ChordGrid is an infrastructure for sensor networks that combines the Chord protocol with grid computing techniques.
    It enables efficient data aggregation and processing across distributed sensor nodes (Zhang et al., 2005).
    \item SensorGrid: SensorGrid leverages Chord's \gls{dht} to provide a scalable and fault-tolerant architecture for sensor networks, allowing for efficient data collection and query processing (Yuan et al., 2005).
\end{itemize}