\abstract{
    This paper explores the Chord framework, a decentralized protocol designed for \gls{p2p} systems, offering an efficient and scalable solution for resource discovery and management in distributed networks.
    We begin with an introduction to distributed systems, emphasizing the evolution towards \gls{p2p} architectures and their importance.
    We then delve into the specifics of the Chord system, explaining its core mechanisms and algorithms, including consistent hashing, finger tables, and the lookup process.
    Following this, we examine various systems and applications built on top of Chord, such as file sharing systems, distributed databases, content distribution networks, and sensor networks, showcasing its versatility and robustness.
    Finally, we conclude with a summary of Chord's contributions to the field of distributed systems and its potential for future developments.
}