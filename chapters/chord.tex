\chapter{Chord System}
The Chord protocol is designed to address the challenges of efficient resource discovery and management in \gls{p2p} systems.
It provides a robust and scalable solution for locating nodes and data in a distributed network.

\section{Architecture of Chord}

Chord organizes nodes in a circular identifier space using consistent hashing.
Each node and data item is assigned a unique identifier, ensuring uniform distribution across the identifier space.

\subsection{Node Identifiers and Consistent Hashing}
Chord uses a consistent hash function, such as SHA-1, to generate identifiers for nodes and data items.
These identifiers are arranged in a circular space, often referred to as an identifier ring.
The position of each node and data item on this ring is determined by their hash value (Stoica et al., 2001).

\subsection{Finger Tables}
Each Chord node maintains a routing table known as a finger table.
The finger table of a node \(n\) contains up to \(m\) entries, where \(m\) is the number of bits in the identifier.
The \(i\)th entry in the finger table of node \(n\) points to the first node that succeeds \(n + 2^{i-1}\) on the identifier circle.
This allows nodes to efficiently route queries by skipping intermediate nodes and reducing the number of hops (Stoica et al., 2001).

\subsection{Lookups and Routing}
When a node needs to locate a data item, it generates the identifier for the item using the hash function.
The node then uses its finger table to forward the query to the appropriate node responsible for the identifier.
This process continues iteratively, with each node forwarding the query closer to the target node, until the node responsible for the data item is found.
The expected number of hops for a lookup is \(O(\log N)\) (Stoica et al., 2001).

\subsection{Joining and Leaving the Network}
When a new node joins the Chord network, it must integrate into the existing structure.
The joining node needs to initialize its finger table and notify other nodes of its presence.
This involves updating finger tables of existing nodes to reflect the new node's position.
When a node leaves the network, it transfers its data to its successor node and notifies other nodes to update their finger tables accordingly.
Chord's protocol ensures that the system remains functional and efficient even as nodes frequently join and leave the network (Stoica et al., 2001).

\section{Example of Chord in Action}
Consider a Chord network with 8 nodes identified by the following hash values: 0, 1, 3, 6, 8, 10, 12, and 15.
Suppose a new node with identifier 7 joins the network.
The steps involved are:
\begin{enumerate}
    \item 1. The new node initializes its finger table.
    \item The new node finds its successor (node 8) and predecessor (node 6).
    \item Node 7 notifies nodes 6 and 8 to update their finger tables.
    \item Data previously managed by node 8 with identifiers between 6 and 7 is transferred to node 7.
\end{enumerate}

Now, if a node wants to lookup data with identifier 11, it will use its finger table to forward the query, minimizing the number of hops until it reaches the responsible node.